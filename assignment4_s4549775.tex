\documentclass[12pt, a4paper]{article}

\usepackage{amssymb}

\setlength\parskip{1em}
\setlength\parindent{0em}

\title{Assignment 4}

\author{Hendrik Werner s4549775}

\begin{document}
\maketitle

\clearpage
\section{Feedback Bram Zandt}
\begin{tabular}{|c|c|}
	\hline
	\textbf{Feedback for} & Bram Zandt\\\hline
	\textbf{Exercise} & 3\\\hline
	\textbf{Feedback from} & Hendrik Werner\\\hline
\end{tabular}

\subsection{Target Audience}
You say that the target audience are mathematicians and computer scientists. I agree with that. Later you say that $\equiv$ and $mod$ are only understandable by computer scientists. This contradicts your earlier point. As far as I know both of these are mathematical symbols and don't have anything to do with Informatics per se.

Your reasoning (apart from that) is sound though, and I agree with your conclusion.

\subsection{Message}
You only discuss RSA being used as a signature algorithm but the article is also concerned with its use to establish public-key cryptosystems. It's a good idea to mention the "comparison" to traditional media. I am not sure if "comparison" is the right word though, as the article wants to take some properties of the old system an apply it to the new one.

\subsection{Effectiveness}
I agree that the abstract is too long and detailed, while the introduction is too short. The point about the "comparison" against traditional media applies here as well.

\clearpage
\section{Feedback Stephanie van Gogh}
\begin{tabular}{|c|c|}
	\hline
	\textbf{Feedback for} & Stephanie can Gogh\\\hline
	\textbf{Exercise} & 3\\\hline
	\textbf{Feedback from} & Hendrik Werner\\\hline
\end{tabular}

You did not include the paper you chose. We were supposed to do this. I was able to find the article with the link you provided after logging in; but the abstract and introduction are not labeled and I did not really know where they are.

At first I thought the gray text was the abstract but that does not appear to be the case, as this switches to the normal, black text mid-sentence.

All in all I was not able to give good feedback for this exercise because the article you chose does not appear to be a scientific article, but an article from a magazine. I could not apply the necessary structure to that article.

\clearpage
\section{Feedback Thijs klein Baltink}
\begin{tabular}{|c|c|}
	\hline
	\textbf{Feedback for} & Thijs klein Baltink\\\hline
	\textbf{Exercise} & 3\\\hline
	\textbf{Feedback from} & Hendrik Werner\\\hline
\end{tabular}

\subsection{Target Audience}
I fully agree with your conclusion what the target audience is; but you did not provide any reasoning for that conclusion. To me it's pretty clear because of all the acronyms used, etc. You should probably mention it anyway.

\subsection{Message}
I agree with the author's message you worked out. You wrote it down very concisely, but still mention all the relevant information. Very well done.

\subsection{Effectiveness}
Again, I agree that the abstract is way too detailed and contains to much information that distracts from the actually important content. However, ironically, this is also true for you answer. I think the whole second paragraph of your answer could have been reduced to 1 or 2 sentences.

\clearpage
\section{Feedback Timo Schrijvers}
\begin{tabular}{|c|c|}
	\hline
	\textbf{Feedback for} & Timo Schwijvers\\\hline
	\textbf{Exercise} & 3\\\hline
	\textbf{Feedback from} & Hendrik Werner\\\hline
\end{tabular}

\subsection{Target Audience}
I mostly agree with what you make out to be the target audience. Apart form the word "efficiently" I cannot find any evidence backing your claim that one needs to know about complexity, however. If this is really that important to the article at a later point, the authors should not have neglected to mention that.

\subsection{Message}
You worked out the same message as I did so I agree with that; but you did so in a very long winded manner. You begin by paraphrasing big chunks of the article, then mention the message.

Probably because you got distracted by that you made the core message a bit vague. My answer would be "The core message is that DNA-computing techniques can be applied to traditionally hard algorithmic problems." or something like that with some evidence.

\subsection{Effectiveness}
I agree with your conclusion and the evidence you provided but again, I think it's unnecessary and distracting to paraphrase the article in between that. This makes your answer longer and buries the important information without a benefit (that I can see).

\clearpage
\section{Feedback Yorick Pelt}
\begin{tabular}{|c|c|}
	\hline
	\textbf{Feedback for} & Yorick Pelt\\\hline
	\textbf{Exercise} & 3\\\hline
	\textbf{Feedback from} & Hendrik Werner\\\hline
\end{tabular}

Your exercise was the best one I had to give feedback for. Really well done $\checkmark$. Therefore I could not add very much, sorry for that.

\subsection{Target Audience}
I agree that the target audience can now be assumed to be more general than initially intended; and with the reasoning given for why the article is still to this new audience.

\subsection{Message}
I have nothing to add, I fully agree.

\subsection{Effectiveness}
I agree with your conclusions. I also think that the article is more difficult to read than necessary because the concepts are explained well but the structure is very strange. I did not make the connection with "stylometrie" myself, and it could be a very good explanation for why this is the case. Very interesting thought.

\end{document}
